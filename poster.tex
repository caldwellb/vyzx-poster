%Introduction
% See: https://rev.cs.uchicago.edu/k4rtik/gemini-uccs
% A fork of https://github.com/anishathalye/gemini

\documentclass[final]{beamer}

% ====================
% Packages
% ====================

\usepackage[T1]{fontenc}
\usepackage{lmodern}
\usepackage[size=a0,orientation=portrait,scale=1.4]{beamerposter}
\usetheme{gemini}
\usecolortheme{uchicago}
\usepackage{graphicx}
\usepackage{booktabs}
\usepackage{tikz}
\usepackage{pgfplots}
\pgfplotsset{compat=1.17}

\usepackage{amsmath,amsfonts}
\usepackage{tikz}
\usetikzlibrary{shapes}
\usetikzlibrary{decorations.pathreplacing}
\usepackage{changepage}
\usepackage{braket}
\usepackage{physics}
\usepackage{mathpartir}
\usepackage{xcolor}
\usepackage{multirow}
\usepackage{tabu}
\usepackage{xspace}
\usepackage{hyperref}
\usepackage[nameinlink]{cleveref}
\usepackage{paralist}
\usepackage[normalem]{ulem}

\usepackage{bussproofs}

\usepackage{soul}

% ====================
% Lengths
% ====================

% If you have N columns, choose \sepwidth and \colwidth such that
% (N+1)*\sepwidth + N*\colwidth = \paperwidth
\newlength{\sepwidth}
\newlength{\colwidth}
\setlength{\sepwidth}{0.025\paperwidth}
\setlength{\colwidth}{0.3\paperwidth}

\newcommand{\separatorcolumn}{\begin{column}{\sepwidth}\end{column}}

% ====================
% Title
% ====================

\title{VyZX: Verifying the ZX Calculus}

\author{Adrian Lehmann \and \textbf{Ben Caldwell} \and Robert Rand}

\institute[shortinst]{University of Chicago}

% ====================
% Footer (optional)
% ====================

\footercontent{
  PLanQC 2022 \hfill
  \href{mailto:caldwellb@uchicago.edu}{caldwellb@uchicago.edu}}
% (can be left out to remove footer)

% ====================
% Logo (optional)
% ====================

% use this to include logos on the left and/or right side of the header:
\logoright{\includegraphics[height=9cm]{logos/cs-logo-white.png}}
\logoleft{\includegraphics[height=9cm]{qr.png}}

% ====================
% Tikz setup
% ====================
\usepackage{tikz}

\usetikzlibrary{backgrounds}
\usetikzlibrary{arrows}
\usetikzlibrary{shapes,shapes.geometric,shapes.misc}
\usetikzlibrary{decorations.pathmorphing}
\usetikzlibrary{decorations.pathreplacing}
\usetikzlibrary{decorations.markings}

\tikzstyle{every picture}=[baseline=-0.25em]

\pgfkeys{/tikz/tikzit fill/.initial=0}
\pgfkeys{/tikz/tikzit draw/.initial=0}
\pgfkeys{/tikz/tikzit shape/.initial=0}
\pgfkeys{/tikz/tikzit category/.initial=0}

\newcommand{\tikzfig}[1]{%
\IfFileExists{#1.tikz}
  {\input{#1.tikz}}
  {%
    \IfFileExists{./figures/#1.tikz}
      {\input{./figures/#1.tikz}}
      {\tikz[baseline=-0.5em]{\node[draw=red,font=\color{red},fill=red!10!white] {\textit{#1}};}}%
  }%
}
\newcommand{\ctikzfig}[1]{%
\begin{center}\rm
  \tikzfig{#1}
\end{center}}

\pgfdeclarelayer{edgelayer}
\pgfdeclarelayer{nodelayer}
\pgfsetlayers{background, edgelayer, nodelayer, main}
\tikzstyle{none}=[inner sep=0mm]
\tikzstyle{every loop}=[]
\tikzstyle{mark coordinate}=[inner sep=0pt,outer sep=0pt,minimum size=3pt,fill=black,circle]
\input{zh.tikzdefs}
\input{zh.tikzstyles}

% ====================
% Packages for Paper
% ====================

\usepackage{mathpartir}
\usepackage{amssymb}
\usepackage{amsmath}

\usepackage{graphicx}

\usepackage{algorithm}
\usepackage{algpseudocode}

\usepackage{hyperref}
% \usepackage[noabbrev,nameinlink]{cleveref}

\usepackage{braket}

\usepackage{wrapfig}
\usepackage{caption}
\usepackage{subcaption}

\usepackage{listings}

% ====================
% Custom Commands
% ====================

\newcommand{\R}{\mathbb{R}}
\newcommand{\C}{\mathbb{C}}
\newcommand{\N}{\mathbb{N}}

\newcommand{\oftype}[2]{\text{#1}\,:\,\text{#2}}
\newcommand{\ZX}[2]{\texttt{ZX}\,\,\text{#1}\,\,\text{#2}}
\newcommand{\nat}{\texttt{nat}}

\newcommand{\VyZX}{\textsl{Vy}\textsc{ZX}\xspace}
\newcommand{\SQIR}{\textsc{sqir}\xspace}
\newcommand{\QASM}{\texttt{QASM}\xspace}
\newcommand{\QLib}{\texttt{QuantumLib}\xspace}
\newcommand{\pyZX}{PyZX\xspace}
\newcommand{\VOQC}{V{\small OQC}\xspace}
\newcommand{\QWIRE}{\texttt{QWIRE}\xspace}
\newcommand{\inQWIRE}{\texttt{inQWIRE}\xspace}
\newcommand{\certiq}{\texttt{CertiQ}\xspace}
\newcommand{\tket}{\texttt{t\(\ket{\text{ket}}\)}\xspace}
\newcommand{\quartz}{Quartz\xspace}

\algnewcommand\algorithmicswitch{\textbf{switch}}
\algnewcommand\algorithmiccase{\textbf{case}}
\algnewcommand\algorithmicassert{\texttt{assert}}

\algdef{SE}[SWITCH]{Switch}{EndSwitch}[1]{\algorithmicswitch\ #1\ \algorithmicdo}{\algorithmicend\ \algorithmicswitch}%
\algdef{SE}[CASE]{Case}{EndCase}[1]{\algorithmiccase\ #1}{\algorithmicend\ \algorithmiccase}%
\algtext*{EndSwitch}%
\algtext*{EndCase}%

\newcommand{\vyzxwire}{\text{\----}}
\newcommand{\vyzxcap}{\subset}
\newcommand{\vyzxcup}{\supset}
\newcommand{\vyzxcompose}{\longleftrightarrow}
\newcommand{\vyzxstack}{\updownarrow}
\newcommand{\vyzxswap}{\times}
\newcommand{\vyzxempty}{\reflectbox{$\varnothing$}}
\newcommand{\Z}[3]{\texttt{Z\_spider\xspace#1\xspace#2\xspace#3}}
\newcommand{\X}[3]{\texttt{X\_spider\xspace#1\xspace#2\xspace#3}}

\newcommand{\hlc}[2][yellow]{{%
    \colorlet{foo}{#1}%
    \sethlcolor{foo}\hl{#2}}%
}

% \newcommand{\notrow}[2]
%   {\begin{figure}[h]
%       \begin{subfigure}{0.3\textwidth}
%         \tikzfig{#1}
%       \end{subfigure}\begin{subfigure}{0.6\textwidth}
%         #2
%         \begin{minipage}
%           \vfill{.1cm}
%         \end{minipage}
%       \end{subfigure}
%     \end{figure}}

% ====================
% Body
% ====================

\begin{document}

\begin{frame}[t, fragile]
\begin{columns}[t]
\separatorcolumn

\begin{column}{\colwidth}

  \begin{block}{Verification and the ZX-Calculus}



\textbf{The Coq Proof Assistant} 

\begin{wrapfigure}{r}{0.2\textwidth}
  \begin{center}
    \includegraphics[width=0.2\textwidth]{logos/coq-logo-medium.png}
  \end{center}
\end{wrapfigure}

  The Coq proof assistant is a proof assistant and programming language for formal verification.
  %
  The methods of formal verification have been used to verify optimizing compilers to guarantee they are bug free.
  %
  The verified quantum circuit optimizer \VOQC\cite{hietala-et-al-2021-VOQC} uses the Coq proof assistant to verify that the optimizations it performs do not change the circuit semantics.

    \textbf{The ZX Calculus} 

  \begin{wrapfigure}{r}{0.4\textwidth}
    \centering
    \begin{subfigure}{0.4\textwidth}
      \centering
      \tikzfig{Z-spider}
%      \subcaption{The Z Spider}\label{fig:Zspider}
    \end{subfigure}
    \begin{subfigure}{0.4\textwidth}
      \centering
      \tikzfig{X-spider}
%      \subcaption{The X Spider}\label{fig:Xspider}
    \end{subfigure}
%    \caption{\centering The Core Objects of the ZX Calculus}
  \end{wrapfigure}


    
    The ZX calculus is a alternative approach to representing quantum programs.
    %
    The objects of the ZX calculus are the \textcolor{green}{Z spiders} (green) and \textcolor{zx_red}{X spiders} (red) to the right, which generalize the quantum mechanical Z and X rotations on qubits.
    %
    They are called ``spiders'' as they can have any number of input or output wires which gives them a spider-like appearence.
    %
    The ZX calculus comes paired with rewrite rules which are known to preserve semantics.
    %
    Tools like PyZX\cite{kissinger2020Pyzx} take advantage of these rules in order to create quantum circuit optimizers.


    \textbf{Verifying ZX}

    Inspired by both \VOQC and \pyZX, we set out to create \VyZX: a verified library for working with ZX diagrams.
    %
    \VyZX is the first step towards creating a fully verified \pyZX-style ZX diagram optimizer.
    %
    We aim to make \VyZX general enough that it can apply to other domains related to the ZX calculus, including lattice surgery and quantum circuit simulation.

  \end{block}
%
  \begin{block}{Drawing Inspiration}

    By looking at how many ways we can construct string diagrams, we have reduced them to a small number of inductive constructors
    \begin{enumerate}
      \item The unit object, which is the empty diagram,
      \item The single wire,
      \item Morphisms, which take $n$ inputs to $m$ outputs,
      \item Braids, which swap two wires,
      \item Sequential composition, which composes two diagrams in sequence , and
      \item Tensor products, which arrange two diagrams in parallel.
    \end{enumerate}

    \begin{figure}
      \centering
      \begin{subfigure}{\textwidth}
      \centering
      \tikzfig{compose_string}
      \qquad
      \tikzfig{parallel_string}
      \vspace{1cm}
      \end{subfigure}
      \begin{subfigure}{\textwidth}
      \centering
        \tikzfig{swap-string}
        \qquad
        \tikzfig{cap}
        \quad
        \tikzfig{cup}
      \end{subfigure}
      \caption{\centering Composition, tensor product, braid, cap, and cup for symmetric monoidal string diagrams.}\label{fig:string}
      \vspace*{-5mm}
    \end{figure}

    To obtain the full ZX calculus we will have two morphisms, the Z and X spider.

  \end{block}

\end{column}

\separatorcolumn

\begin{column}{\colwidth}

  \begin{alertblock}{Inductively Drawing Diagrams}

    \begin{figure}[t]
      \centering
      \begin{subfigure}{\textwidth}
        \inferrule
          {\oftype{in out}{\(\N\)} \\ \oftype{\(\alpha\)}{\(\R\)}}
          {\oftype{\texttt{Z Spider} in out \(\alpha\)}{\ZX{in}{out}}}
        \end{subfigure}
        
      \begin{subfigure}{\textwidth}
        \inferrule
          {\oftype{in out}{\(\N\)} \\ \oftype{\(\alpha\)}{\(\R\)}}
          {\oftype{\texttt{X Spider} in out \(\alpha\)}{\ZX{in}{out}}}   
        \end{subfigure}
    
      \begin{subfigure}{0.5\textwidth}
        \inferrule{ }{\oftype{\texttt{Cap}}{\ZX{0}{2}}}
      \end{subfigure}\begin{subfigure}{0.5\textwidth}
        \inferrule{ }{\oftype{\texttt{Cup}}{\ZX{2}{0}}}
      \end{subfigure}
      
      \begin{subfigure}{0.5\textwidth}
        \inferrule{ }{\oftype{\texttt{Swap}}{\ZX{2}{2}}}
      \end{subfigure}\begin{subfigure}{0.5\textwidth}
        \inferrule{ }{\oftype{\texttt{Empty}}{\ZX{0}{0}}}
      \end{subfigure}
      \begin{subfigure}{\textwidth}
        \centering
      \inferrule
        {\oftype{zx1}{\ZX{in}{mid}} \\ \oftype{zx2}{\ZX{mid}{out}}}
        {\oftype{\texttt{Compose} zx1 zx2}{\ZX{in}{out}}}
      \end{subfigure}
      \begin{subfigure}{\textwidth}
      \inferrule
        {\oftype{zx1}{\ZX{in1}{out1}} \\ \oftype{zx2}{\ZX{in2}{out2}}}
        {\oftype{\texttt{Stack} zx1 zx2}{\ZX{(in1 + in2)}{(out1 + out2)}}}
      \end{subfigure}
      \caption{The inductive constructors for block representation ZX diagrams}\label{fig:blockconstructors}
    \end{figure}

    From these constructors we then define Wire as a notation for $\Z{1}{1}{0}$.

    \begin{figure}
      \begin{tabular}{| c | c |}
        \hline
        Constructor & Notation \\
        \hline
        Empty & $\vyzxempty$ \\
        Cap & $\vyzxcap$ \\
        Cup & $\vyzxcup$ \\
        Swap & $\vyzxswap$ \\
        Compose & $\vyzxcompose$ \\
        Stack & $\vyzxstack$ \\
        Wire & $\vyzxwire$ \\
        \hline
      \end{tabular}
      \caption{Coq notations for our diagrammatic constructors.}
      \vspace*{-5mm}
    \end{figure}

  \end{alertblock}

  \begin{block}{Diagram Semantics}

    To give our inductive diagrams a semantics we use the verified library $\QLib$ and transform our diagrams into $\QLib$ matrices.

    \begin{figure}
      \tikzfig{diagram-semantics}
      \caption{The \texttt{semantics} function for giving semantics to inductively defined ZX diagrams.}
    \end{figure}

    Using these semantics we define a relation $\propto$ and say that $zx_1 \propto zx_2$ if and only if
    \[
      \exists c \in \mathbb{C} : c \neq 0 \land \texttt{sem}(zx_1) = c \cdot \texttt{sem}(zx_2)
    \]
    this relation will allow us to define, prove, and use the ZX calculus rewrite rules.

  \end{block}

  % \begin{block}{Equivalence of Diagrams}
  %
  %   % We cannot just consider two ZX diagrams equivalent if they are syntactically equivalent but rather if they are semantically equivalent.
  %   Trivially two syntactically equal diagrams are to be considered equal.
  %   %
  %   For making useful statements, however, we require a notion of semantic equivalence.
  %   %
  %   In the ZX calculus, though, we only care about equivalence up to multiplication by a constant factor, as rules will introduce constant factors and we are able to rebuild any constant factor if necessary using ZX constructions~\cite{vandewetering2020zxcalculus}.
  %   
  %   Within \VyZX we define a relation called \textit{proportional} and give it the notation $\propto$.
  %   %
  %   \[
  %     zx_1 \propto zx_2 \iff  \exists c \in \mathbb{C} : c \neq 0 \land \texttt{semantics}(zx_1) = c \cdot \texttt{semantics}(zx_2)
  %   \]
  %   %
  %   \texttt{ZX\_semantics} $zx_1$ = $c * {}$ \texttt{ZX\_semantics} $zx_2$.
  %   %
  %   We prove that $\propto$ is an equivalence relation as we might expect.
  %   % 
  %   We then prove that our composition operators respect proportionality: That is, if $zx_1 \propto zx_1'$ and $zx_2 \propto zx_2'$ then $\texttt{Compose}~zx_1~zx_2 \propto \texttt{Compose}~zx_1'~zx_2'$ and $\texttt{Stack}~zx_1~zx_2 \propto \texttt{Stack}~zx_1'~zx_2'$. We add this fact to Coq as a \emph{parametric morphism}, allowing us to rewrite using our equivalences even inside a broader diagram.
  %   %
  %   With proportionality defined, we proceed to verify different rewrite rules within the ZX calculus by proving their diagrams to be proportional.
  % \end{block}


\end{column}

\separatorcolumn

\begin{column}{\colwidth}

  \begin{block}{Example: Untangling Nots}

    \begin{figure}
        \begin{subfigure}[b]{\textwidth}
            $  \texttt{Definition CNOT\_R} := $
            \begin{align*}
              (\Z{1}{2}{0} \vyzxstack \vyzxwire) &\vyzxcompose \hspace{5.2cm} \\ 
              (\vyzxwire \vyzxstack \X{2}{1}{0}) &
            \end{align*}
            \vspace*{.5mm}
        \end{subfigure}
        \begin{subfigure}[b]{.3\textwidth}
            \tikzfig{cnot-r}
        \end{subfigure}
    \end{figure}

    \begin{figure}
        \begin{subfigure}[b]{\textwidth}
            $  \texttt{Definition CNOT\_L} := $
          \begin{align*}
            (\vyzxwire \vyzxstack \X{2}{1}{0}) &\vyzxcompose  \hspace{5.2cm} \\
            (\Z{1}{2}{0} \vyzxstack \vyzxwire)
          \end{align*}
          \vspace*{.5mm}
        \end{subfigure}
        \begin{subfigure}[b]{.3\textwidth}
            \tikzfig{cnot-l}
        \end{subfigure}
    \end{figure}

    \(
    \texttt{Lemma CNOT\_R\_PROP\_L : } \)

    \qquad\(
    \texttt{sem}(\texttt{CNOT\_R}) = \texttt{sem}(\texttt{CNOT\_L})
    \)
    \begin{figure}[h]
      \tikzfig{cnot-prop}
    \end{figure}

    Performing the final reduction step and turning the diagrams into matrices we get for the left hand side
    \[
      (\begin{bmatrix} 
         1 & 0 \\ 
         0 & 1 
       \end{bmatrix} 
       \otimes 
       (H 
       \times 
       \begin{bmatrix} 
         1 & 0 & 0 & 0 \\
         0 & 0 & 0 & 1 
       \end{bmatrix} 
       \times 
       (H \otimes H)))
       \times 
      (\begin{bmatrix}
         1 & 0 \\
         0 & 0 \\
         0 & 0 \\
         0 & 1
       \end{bmatrix}
       \otimes
       \begin{bmatrix}
         1 & 0 \\
         0 & 1 
       \end{bmatrix})
     \]
     \begin{center}{ and for the right hand side }\end{center}
     \[
       (\begin{bmatrix} 
         1 & 0 & 0 & 0 \\
         0 & 0 & 0 & 1 
       \end{bmatrix} 
       \otimes
       \begin{bmatrix}
         1 & 0 \\
         0 & 1 
       \end{bmatrix})
       \times (
       \begin{bmatrix}
         1 & 0 \\
         0 & 1 
       \end{bmatrix}
       \otimes
       ( (H \otimes H) \times
       \begin{bmatrix}
         1 & 0 \\
         0 & 0 \\
         0 & 0 \\
         0 & 1
       \end{bmatrix}
       \times H))
       )
     \]
     Using the proof automation available to \QLib, this equality can be solved automatically, and will look like this:

\begin{lstlisting}
Lemma CNOT_R_EQ_L : 
    ZX_semantics ZX_CNOT_R = 
    ZX_semantics ZX_CNOT_L.
Proof.
  simpl.
  rewrite wire_identity_semantics.
  unfold_spider.
  autorewrite with Cexp_db.
  solve_matrix.
Qed.
\end{lstlisting}

% An identical proof can then be written to verify that 
% \begin{lstlisting}
% Lemma CNOT_R_CORRECT :
%     ZX_semantics ZX_CNOT_R = cnot
% \end{lstlisting}
% using the cnot gate from \QLib.
% \begin{lstlisting}
% Lemma ZX\_CNOT\_equiv : ZX\_semantics ZX\_CNOT\_l = ZX\_semantics ZX\_CNOT\_r.
% Proof.
%     simpl.
%     rewrite wire\_identity\_semantics.
%     unfold\_spider.
%     autorewrite with Cexp\_db.    
%     solve\_matrix.
% Qed.
% \end{lstlisting}
      

  \end{block}
%   \begin{block}{Current Status}
%
%     % \paragraph*{Common gates}
%     We translated common gates from the circuit model to the ZX-calculus and proved their semantic correctness.
%     
%     \begin{figure}
%       \centering
%       \tikzfig{gates}
%       \caption{\centering Quantum gates represented in the ZX calculus. Note that the $H$ node omits the its label and is common to ZX diagrams as syntactic sugar for the rotations above.}\label{fig:gates}
%     \end{figure}
%
%
%     \begin{columns}
%       \begin{column}{0.53\colwidth}
%         \justify
%         We proved the commutation of stack and compose with our language, an essential tool for most proof.
%         %
%         Another proof tool we created handles automatically swapping the colors of diagrams.
%         %
%         To accomplish this we first color swapped individual spiders by surrounding them with Hadamard gates.
%         We then use Coq's induction tactics, cover the other base cases, and define $\odot$ as a function which swaps a diagram's colors.
%         \quad\\
%
%         The bialgebra rule, while not intuitive, is crucial for many ZX proofs as it allows for the rearranging of edges between nodes ~\cite{vandewetering2020zxcalculus}.
%         %
%         This proof ends up being computationally expensive to check, but it is a very necessary tool.
%
%         This incomplete form of the Hopf rule is a natural description for the inductive language defined here, and simply is 
%         \[
%           \Z{1}{2}{} \vyzxcompose \X{2}{1}{} \propto 
%         \] 
%         \[
%           \Z{1}{0}{} \vyzxcompose \X{0}{1}{}.
%         \]
%
%         \begin{figure}
%             \centering
%             \tikzfig{hopf}
%             \caption{\centering The Hopf rule}\label{fig:hopf}
%         \end{figure}
%
%       \end{column}
%       \begin{column}{0.37\colwidth}
%
%         \begin{figure}
%           \centering
%           \tikzfig{stack-compose-comm}
%           \caption{\centering The commutation of stack and compose}
%           \label{fig:stack-compose-comm}
%         \end{figure}
%         
%         \begin{figure}
%             \centering
%             \tikzfig{bihadamard}
%             \caption{\centering Swapping a spider's color using a bi-Hadamard construction}
%             \label{fig:bihadamard}
%         \end{figure}
%
%         \begin{figure}
%             \centering
%             \tikzfig{bihadamard-diag}
%             \caption{\centering Swapping colors of a diagram using a bi-Hadamard construction}
%             \label{fig:bihadamard-diag}
%         \end{figure}
%
%         \begin{figure}
%           \centering
%           \tikzfig{bi-alg}
%           \caption{\centering The bialgebra rule}\label{fig:bi-alg}
%         \end{figure}
%
%         \begin{figure}
%           \tikzfig{spider-fusion-1-1}
%           \caption{\centering Spider fusion rule}
%         \end{figure}
%
%       \end{column}
%
%     \end{columns}
%
% One of the most important rules is that spiders connected by an arbitrary (non-zero) number of edges can be fused into a single node with the angles added~\cite{vandewetering2020zxcalculus}.
% %
% Further, the reverse is true: any spider can be split such that the two new spiders add to the original angle.
% A corollary of this is that we can split phase-gadgets off nodes.
%
% Since this rule fundamentally works based on adjacency rather than block representation construction, we have not fully implemented it at the time of writing.
% We do have a restricted version proven where two nodes are just connected to each other by a single wire, which will form the basis of the general spider fusion proof.
% This allows us to use \QLib's algebraic rewrites of complex matrices to combine the angles easily.
% Given the algebraic rewrite this rule is computationally very efficient.
%
%   \end{block}

  \begin{block}{References}

    {\small \bibliographystyle{plain}\bibliography{poster}}

  \end{block}

\end{column}

\separatorcolumn
\end{columns}
\end{frame}
\end{document}
